\documentclass[12pt]{article}

\usepackage{listings}
\usepackage{color}
\usepackage{xcolor}
\usepackage{caption}

\lstset{language=Java,captionpos=t,tabsize=4,frame=no,keywordstyle=\color{blue},
   commentstyle=\color{gray},stringstyle=\color{red},numbers=left,numberstyle=\scriptsize,
   numbersep=5pt,breaklines=true,showstringspaces=false,basicstyle=\footnotesize,emph={label}}
   
\DeclareCaptionFont{white}{\color{white}}
\DeclareCaptionFormat{listing}{\colorbox{darkgray}{\parbox[c][0.3cm]{\textwidth}{#3}}}
\captionsetup[lstlisting]{format=listing,labelfont=white,textfont=white}

\title{
{\bf Fantasy League Advanced\\
Imperative Language (FLAIL)}\\
{\small A Project Proposal for {\it COMS W4115: Programming Language \& Translators}}
}

\author{
	Stephanie Aligbe\\
	\texttt{\small{sna2111@columbia.edu}}
	\and
	Elliot Katz\\
	\texttt{\small{epk2102@columbia.edu}}
	\and
	Tam Le\\
	\texttt{\small{tvl2102@columbia.edu}}
	\and
	Dillen Roggensinger\\
	\texttt{\small{der2127@columbia.edu}}
	\and
	Anuj Sampathkumaran\\
	\texttt{\small{as4046@columbia.edu}}
}

\date{February 23, 2011}

\begin{document}
\maketitle

\begin{abstract}
We present FLAIL, a new programming language designed to facilitate the creation of fantasy league games and simulations. 
\begin{center}

\end{center}
\end{abstract}

\section{Introduction}
FLAIL is an imperative, object-oriented programming language.

\section{Features \& Syntax}

\section{Sample Code}
\begin{lstlisting}[language=Java]
public class Application {
   public static void main(String[] args) {
      System.out.println("Hello World!");
   }
}
\end{lstlisting}

\end{document}
