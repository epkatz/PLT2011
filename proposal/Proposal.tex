\documentclass[12pt]{article}

\usepackage{listings}
\usepackage{color}
\usepackage{xcolor}
\usepackage{caption}

\lstset{language=Java,captionpos=t,tabsize=4,frame=no,keywordstyle=\color{blue},
   commentstyle=\color{gray},stringstyle=\color{red},numbers=left,numberstyle=\scriptsize,
   numbersep=5pt,breaklines=true,showstringspaces=false,basicstyle=\footnotesize,emph={label}}
   
\DeclareCaptionFont{white}{\color{white}}
\DeclareCaptionFormat{listing}{\colorbox{darkgray}{\parbox[c][0.3cm]{\textwidth}{#3}}}
\captionsetup[lstlisting]{format=listing,labelfont=white,textfont=white}

\title{
{\bf Fantasy League Advanced\\
Imperative Language (FLAIL)}\\
{\small A Project Proposal for {\it COMS W4115: Programming Language \& Translators}}
}

\author{
	Stephanie Aligbe\\
	\texttt{\small{sna2111@columbia.edu}}
	\and
	Elliot Katz\\
	\texttt{\small{epk2102@columbia.edu}}
	\and
	Tam Le\\
	\texttt{\small{tvl2102@columbia.edu}}
	\and
	Dillen Roggensinger\\
	\texttt{\small{der2127@columbia.edu}}
	\and
	Anuj Sampathkumaran\\
	\texttt{\small{as4046@columbia.edu}}
}

\date{February 23, 2011}

\begin{document}
\maketitle

\begin{abstract}
We present FLAIL, an imperative, object-oriented programming language designed to facilitate the creation of fantasy league games and simulations. 
\begin{center}

\end{center}
\end{abstract}

\section{Introduction \& Motivation}
As defined on Wikipedia, ``a fantasy sport (also known as {\bf rotisserie}, {\bf roto}, or {\bf owner simulation}) is a game where participants act as owners to build a team that competes against other fantasy owners based on the statistics generated by the real individual players or teams of a professional sport."
\\\\
The popularity of fantasy sports has exploded in recent years. A recent study in 2007 by the {\it  Fantasy Sports Trade Association (FSTA)} estimated that nearly 30 million people in the U.S. and Canada, ranging in age from 12 and above, participated and played in organized fantasy sports leagues. The study also estimated the spending habits and overall economic impact of fantasy sports players. Consumers engaging in this ever expanding hobby spent \$800 million directly on fantasy sports products and an additional \$3 billion worth of related media products (such as DirecTV's NFL Sunday Ticket and XM Radio's coverage of all MLB baseball games).
\\\\
The growth of this industry has been dramatic. During the early 1990s, there was an estimated 3 million people in North America playing fantasy sports. A similar FSTA study in 2003 showed that number had swelled to 15 million. This impressive growth looks to continue indefinitely as this latest FSTA study estimates teenagers in both the U.S. and Canada play fantasy sports at a higher rate than the national average, with 13 percent of teens paying in the U.S. and 14 percent playing in Canada.
\\\\
This burgeoning popularity for fantasy sports is not, however, restricted to North America alone. For example, a 2008 study by a European-based market research company estimated the number of fantasy sports players in Britain to range between 5.5 and 7.5 million and varying in age between 16-64, of which 80 percent participated in fantasy soccer.
\\\\
As students of computer sciences and via the personal experiences of several team members who have played in fantasy sports leagues in the past, we are thus motivated to design and create a programming language that will...

\section{Features \& Syntax}

\section{Sample Code}
\begin{lstlisting}[language=Java]
public class Application {
   public static void main(String[] args) {
      System.out.println("Hello World!");
   }
}
\end{lstlisting}

\end{document}
